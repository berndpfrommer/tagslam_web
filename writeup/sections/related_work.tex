\section{Related Work}
As a review of the vast SLAM literature is beyond the scope of this
paper, we refer to the thorough survey in Ref.\ \cite{cadena2016}. In
this section we will focus on SLAM that utilizes fiducial markers.

Ref. \cite{neunert2016} presents a visual-inertial approach for
obtaining ground truth positions from a combination of inertial
measurement unit (IMU) and camera. They also utilize April\-Tags, but in
contrast to TagSLAM use an Extended Kalman Filter (EKF), a method that
is well suited for their goal to effectively provide a real-time,
outdoor, low-cost motion capture system. Since TagSLAM uses a graph
optimizer instead of a filter, it can leverage observations from both
the past and the future, resulting in 
smoother trajectories. In case IMU data is available, TagSLAM can be set up
to operate in a similar fashion to the system described in
\cite{neunert2016} by supplying externally generated VIO data.

Most similar to our work is Ref.\ \cite{munoz2018}, where a system is
described that can perform real-time SLAM from square planar markers
(SPM-SLAM). By using keyframes and having different algorithms for
tracking and loop closure, they achieve high performance and
scalability to large scenes. In
contrast, TagSLAM considers every frame a keyframe, and relies on
iSAM2 \cite{kaess2011} to exploit sparsity. This significantly simplifies
TagSLAM's code base, and facilitates the implementation of its
flexible model.
For long trajectories and many tags, we expect SPM-SLAM to scale
better due to its use of keyframes. However, our
experiments (Sec. \ref{sec:application}) indicate that TagSLAM can handle
scenes that are sufficiently large for many situations. Unlike
TagSLAM, SPM-SLAM cannot incorporate keypoints via odometry
measurements, although this shortcoming is addressed in a forthcoming
paper (``UcoSLAM''). As far as robustness is concerned, both TagSLAM and
SPM-SLAM rely on detecting tag pose ambiguity \cite{schweighofer2006},
but SPM-SLAM attempts to use the measurements right away by relying on
a sophisticated two-frame initialization algorithm, whereas TagSLAM delays use
of the tag's observations until its pose can be determined
unambiguously. For achieving robustness during relocalization,
SPM-SLAM relies again on pose ambiguity measures, whereas TagSLAM
additionally considers the tag's apparent size.

Also closely related to the present paper is the underwater SLAM performed
in \cite{westman2018}. The authors leverage AprilTags and the
GTSAM factor graph optimizer to obtain vision based ground truth poses
and extrinsic calibration. The focus of their work however is more on
providing a solution for a particular problem, whereas TagSLAM aims to
be a general purpose framework that can be easily applied to many
different settings. In fact, all the factors in \cite{westman2018}
that are related to AprilTags are already implemented in TagSLAM. The
code structure of TagSLAM is designed such that adding the
problem-specific XHY and ZPR factors from \cite{westman2018} should be
straight forward.



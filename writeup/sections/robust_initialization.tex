\section{Robust Initialization}
\label{sec:robustinit}
Nonlinear non-convex optimizers such as GTSAM \cite{kaess2011} are
iterative solvers and hence rely on a starting guess that is
reasonably close to the optimum. If for example a camera pose is
initialized such that one of the observed tag corner points lies
behind the camera, the optimizer will likely fail.

Several sources of error can contribute to poor pose
initialization. TagSLAM has been used for several projects already,
and in our experience, human errors are frequently the root cause,
such as inaccuracies or outright typos when entering measured tag
poses, intrinsic or extrinsic calibration errors, duplicate tag ids,
errors in tag size due to printing or misspecfication, using tags that
are not planar, or supplying unsychronized stereo images. One of the
design goals of TagSLAM was to produce, as much as possible, a
reasonable result with a bounded error even with compromised input
data, such that at least the scene can be visualized for further
analysis.

Even when all input data is correct, initializing a camera or tag pose
can be challenging, for example because the corner of the tag is not
detected accurately, which may happen under motion blur, or when a tag
is partially occluded.  As described in Ref.\ \cite{jin2017}, there
are two camera poses from where a single tag looks quite similar, with
only perspective distortion distinguishing between them. For a tag
that is barely large enough to be detected and is viewed at a shallow
angle, a tag corner error of just a single pixel can lead to a
dramatically different pose initialization. Localizing off of a single
tag is therefore intrinsically difficult, and should be
avoided. Satisfactory results from a single tag can only be expected
in combination with odometry input or when the tag image is
sufficiently large.

When multiple cameras and several tags are involved with known or
unknown poses it is anything but obvious which measurements to use,
and in which order.  For example, should the pose of camera 1 be
established first from the tag corners, then the pose of camera 2 via
a known extrinsic calibration, or the other way round? Which tags
should be used for this purpose if several are visible?

In case the tag poses are known, one might be tempted to answer the
last question with: use the corner points of all tags simultaneously
with a perspective n-point method \cite{fischler1981} (PnP). In
practice we find that PnP is not robust to misspecified tag
poses. Moreover when it fails it is not clear which of the tags is in
error, necessitating an expensive elimination process. For this reason
we base all pose initialization on homographies \cite{ma2003} from a
minimum set of tags only, carefully picking which tags to use, and in
which order. The remainder of this section will describe how exactly
this is done.

To fully exploit the history of observations, two separate graphs are
maintained: the {\em full graph} contains factors and variables
pertaining to all measurements up to the current time, whereas the
{\em optimized graph} only has those variables and factors that are
sufficiently constrained to form a well conditioned optimization
problem. All incoming measurements are thus entered immediately into
the {\em full graph}, but only find their way into the {\em optimized
  graph} when the variables can actually be initialized.

Usually several static poses can be initialized right away because a
prior is available. In Fig.\ \ref {fig:sample_graph} for example, the
poses $\ctrans{l}{T}{2}$, $\ctrans{\mathrm{w}}{T}{l}$,
$\ctrans{\mathrm{r}}{T}{\mathrm{c}}$, and
$\ctrans{\mathrm{b}}{T}{105}$ are determinable due to the prior
factors shown to their left, and can therefore be directly inserted
into the {\em optimized graph}.

\subsection{Subgraph discovery}
As measurements arrive, they give rise to new factors that create
edges in the graph between existing and new variables. Variables
connected to these factors may be rendered determinable, which in
turn, through existing factors, may cause other variables to become
determinable as well. Thus every new factor can give rise to a
subgraph of newly determinable variables. We refer as {\em subgraph
  discovery} to the exhaustive traversing of the graph until no more
new variables can be determined. During this process, all
deteriminable variables and factors are entered into the
subgraph. Further, an {\em initialization list} is created that
contains the factors in the order they are discovered. The
initialization list later governs the order in which the subgraph will
be initialized.  The discovered subgraph depends on the new factor
from which the discovery is started, so potentially, each new factor
gives rise to a different subgraph. In practice the subgraphs are
connected, and often a set of new factors generates only a single
subgraph.

Note that also factors arising from past measurements may enter the
subgraph. For example, if only tags $A$ and $B$ with unknown pose were
seen at time step $t$, no camera rig pose could be determined. But if
at $t+1$ tags $B$ and $C$ are observed, and tag $C$ has known pose,
then the subgraph at $t+1$ will contain poses for tags $A$, $B$, and
$C$, although tag $A$ was observed in the previous step.

All variables of the so generated subgraph are determinable, but if
any dynamic poses from previous time steps are present, they must be
constrained or the problem is ill determined. This is done by
inserting absolute priors for those
poses. Fig.\ \ref{fig:sample_subgraph} shows a subgraph, derived from
the graph in Fig.\ \ref{fig:sample_graph} during time step $t+1$. Note
the removal of the tag projection factors and of the block pose for
time $t$, and the addition of a prior (colored red) on pose
$\ctrans{\mathrm{w}}{T}{\mathrm{r}}(t)$.

\begin{figure}[ht]
  \newcommand{\relfacgraphsize}{0.75}
  \begin{center}
    \begin{tabular}{cc}
      \resizebox{\relfacgraphsize\columnwidth}{!}{
      \begin{tikzpicture}

\newcommand{\ysepstatic}{0.2cm}     % y sep static poses
\newcommand{\xsepprior}{0.2cm}      % x sep of priors
\newcommand{\xsepdyn}{1.6cm}        % x sep dyn pose
\newcommand{\xsepfac}{1.1cm}        % x sep dyn pose
\newcommand{\nsize}{16mm}        % minimum node size
\newcommand{\facoff}{12mm}        % factor offset
\newcommand{\xsepfactpo}{4.0cm}        % x sep dyn pose t+1
\newcommand{\xseprelfac}{0.75cm}        % rel pose prior factor
\tikzstyle{node} = [latent,minimum size=\nsize];

%
%
% The static poses
%
\node[node]                               (lab_T_2)     {$\ctrans{l}{T}{2}$};
\node[node, below=\ysepstatic of lab_T_2]   (w_T_lab)     {$\ctrans{\mathrm{w}}{T}{l}$};
\node[node, below=\ysepstatic of w_T_lab]   (rig_T_cam)   {$\ctrans{\mathrm{r}}{T}{\mathrm{c}}$};
\node[node, below=\ysepstatic of rig_T_cam] (block_T_105) {$\ctrans{\mathrm{b}}{T}{105}$};

%
%
% priors on static poses
%
\factor[left=\xsepprior of lab_T_2]     {pAtag2} {above:{$p_A$}} {lab_T_2} {};
\factor[left=\xsepprior of w_T_lab]     {pAlab}  {above:{$p_A$}} {w_T_lab} {};
\factor[left=\xsepprior of rig_T_cam]   {pAcam}  {above:{$p_A$}} {rig_T_cam} {};
\factor[left=\xsepprior of block_T_105] {pA105}  {above:{$p_A$}} {block_T_105} {};

%
% dynamic poses at t
%
\node[node,right=\xsepdyn of lab_T_2]     (w_T_rig_t)   {$\ctrans{\mathrm{w}}{T}{\mathrm{r}}(t)$};
\factor[color=red,left=\xsepprior of w_T_rig_t]  {pwTrigt}  {above:{$p_A$}} {w_T_rig_t} {};

%
% dynamic poses at t+1
%
\node[node,right=\xsepdyn of w_T_rig_t]   (w_T_rig_tp1)   {$\ctrans{\mathrm{w}}{T}{\mathrm{r}}(t+1)$};
\node[node,right=\xsepdyn of w_T_block_t] (w_T_block_tp1) {$\ctrans{\mathrm{w}}{T}{\mathrm{b}}(t+1)$};

%
% tag projection factors at t + 1
%
\factor[right=\xsepfactpo of w_T_lab]   {pT2tp1}  {above:{$p_T$}} {lab_T_2,w_T_lab,rig_T_cam,w_T_rig_tp1}{};
\factor[right=\xsepfactpo of rig_T_cam] {pT105tp1}{below:{$p_T$}} {block_T_105,w_T_rig_tp1,rig_T_cam,w_T_block_tp1}{};

%
% relative pose priors
%

\factor[right=\xseprelfac of w_T_rig_t]  {prelrig}{above:{$p_R$}} {w_T_rig_t,w_T_rig_tp1}{};

%\factor[right=\xseprelfac of w_T_block_t] {prelblock}{above:{$p_R$}} {w_T_block_t,w_T_block_tp1}{};

\end{tikzpicture}

      }
    \end{tabular}
  \end{center}
  \caption{Subgraph of graph in Fig.\ \ref{fig:sample_graph}}
  \label{fig:sample_subgraph}
\end{figure}

To arrive at a complete set of subgraphs, all new factors are entered
into the {\em new factor list}. The order of the factors is important,
and will be discussed further below. An iteration over this list is
performed, and, unless the new factor is already part of an already
discovered subgraph, a new subgraph is generated by discovery. This
procedure results in a set of disjoint subgraphs with all determinable
variables and connected factors, as well as the corresponding {\em
  initialization lists}.

\subsection{Order of subgraph discovery}

What still remains to be settled is the order in which factors are
entered into the {\em new factor list}. This strongly affects the {\em
  initialization lists} generated during subgraph discovery, and hence
the order in which measurements are used for initialization. For
instance in Fig.\ \ref{fig:sample_subgraph}, there are two ways to
initialize $\ctrans{\mathrm{w}}{T}{\mathrm{r}}(t+1)$. One is through
the relative pose factor with respect to
$\ctrans{\mathrm{w}}{T}{\mathrm{r}}(t)$, the other through the tag
projection factor on tag 2.

By examining the typical failure cases of several alternative
approaches, the following order for entering factors into the {\em new
  factor list} was established:
\begin{enumerate}[wide, labelwidth=!]
\item Any relative pose priors. Note that this prefers odometry
  updates, which are typically more reliable in establishing a body
  pose than initializing it from tag observations.
\item Any tag projection factors. These factors are sorted in
  descending order by pixel area of the observed tag, irrespective
  which camera they were observed from. This means that larger tags
  will be used first to establish a pose.
\item Any factors that do not establish a pose. Examples are problem
  specific additional factors such as distance measurements between
  tag corners.
\end{enumerate}

\subsection{Optimization}
Once the subgraphs have been obtained, their variables are initialized
in the order specified by the corresponding {\em initialization
  list}. Then all subgraphs are optimized using GTSAM, in
non-incremental mode, i.e. without using iSAM2 \cite{kaess2011}, and
their error is evaluated. This step is analogous to model validation
in RANSAC \cite{fischler1981}. If a subgraph's error falls below a
configurable threshold, the subgraph is accepted, and its factors and
optimized values are transferred to the {\em optimized
  graph}. Subsequently, the {\em optimized graph} is optimized with an
iSAM2 update step.

In the rare case where a subgraph is rejected due to excessive error,
an initialization with different ordering is attempted. Since
exhaustively trying all possible orderings is computationally too
expensive, an ad hoc procedure is adopted that was found to work
satisfactorily in practice: the {\em initialization list} of the
subgraph is rotated such that the first factor goes to the end of the
list, and all other elements of the list advance by one. This implies
that successively smaller tags are used to seed the initialization
process. The subgraph is initialized with the new ordering, followed
by optimization. In case the subgraph error is still too high, this
process is repeated until the original ordering is reached again. If
no initialization ordering leads to an acceptable error, the subgraph
is rejected, thus preventing an outlier measurement from contaminating
the {\em optimized graph}.

\subsection{Diagnostics}
Rejection of a subgraph is usually a strong indicator of faulty input
parameters or misdetected tag corners. In most cases however, subgraph
rejection is not fatal, and TagSLAM will successfully handle
subsequent incoming data. For diagnosis, TagSLAM produces per-factor
and time resolved error statistics. Such output is essential for
tracking down e.g. incorrectly specified tag poses.

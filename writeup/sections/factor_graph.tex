\section{Factor Graph}
Our SLAM is formulated as a bipartite factor graph with two types of nodes:
the variables (poses) which are elements of the set $\Theta$, and the
factors, which constrain the variables via the set of measurements  $\mathcal{Z}$. The
factor graph defines a probability $P(\Theta|\mathcal{Z})$ that
assumes its maximum a posteriori (MAP) value for the optimal variable set
$\Theta^*$, given the measurements:
\begin{equation}
\Theta^* = \argmax_\Theta P(\Theta|\mathcal{Z})\ .
\end{equation}
The set of variables $\mathcal{Z}$ contains:
\begin{itemize}
\item The camera poses $\ctrans{\mathrm{body}(\mathrm{cam}\ j)}{T}{\mathrm{cam}\ j}$, with respect to the bodies they are attached to.
\item The tag poses $\ctrans{\mathrm{body}(\mathrm{tag}\ k)}{T}{\mathrm{tag}\ k}$,  relative to their respective bodies.
\item The world poses $\ctrans{\mathrm{w}}{T}{\mathrm{body}\ l}$, of static bodies.
\item The time-dependent world poses $\ctrans{\mathrm{w}}{T}{\mathrm{body}\ m}(t)$, $t = 1\dots N_{\mathrm{t}}$ of dynamic bodies.
\end{itemize}
The likelihood is expressed \cite{kaess2011} as a product of factors $p^{(i)}$
that connects the variables with each other via measurements to form the desired graph structure:
\begin{equation}
  \label{eq:totalprob}
  P(\Theta|\mathcal{Z}) = \prod_ip^{(i)}(\Theta|\mathcal{Z})\ .
\end{equation}
To make the factors $p^{(i)}$ computationally tractable, we follow the standard
approach \cite{kaess2011} and model them as Gaussians:
\begin{equation}
  g(x;\mu,\Sigma) = \exp(-\frac{1}{2}||x\ominus\mu||_\Sigma^2)\ .
\end{equation}
Here, $x$ is the variable, $\mu$ the center of the Gaussian, and $\Sigma$ defines the Mahalanobis distance.
Note the use of the $\ominus$ operator, which reduces to straight subtraction for elements of a vector space,
but produces 6-dimensional Lie algebra coordinates when applied to elements on the SE(3) manifold:
\begin{equation}
  \label{eq:ominus}
  \ctrans{}{T}{A} \ominus \ctrans{}{T}{B} =
  \left[[\log(\mathrm{Rot}(\ctransi{}{T}{B}\ctrans{}{T}{A}))]_\vee^\top,\mathrm{Trans}(\ctransi{}{T}{B}\ctrans{}{T}{A})^\top\right]^\top\ .
\end{equation}
In (\ref{eq:ominus}) Rot() and Trans() refer to the rotational and translational part of the SE(3) transform,
respectively, log() is the matrix logarithm, and $\vee$ denotes the
{\em vee} map operator. Equipped with the definition of a Gaussian
on SE(3), we can now introduce the basic factors $p^{(i)}$ from Eq.\ (\ref{eq:totalprob}).

{\em Absolute Pose Prior}. This unary factor can be used to specify a prior pose
$\ctrans{}{T}{0}$ with noise $\Sigma$ for e.g. a tag or a camera:
\begin{equation}
  p_A(\ctrans{}{T}{}|\ctrans{}{T}{0},\Sigma) = g(\ctrans{}{T}{}; \ctrans{}{T}{0}, \Sigma)\ .
\end{equation}

{\em Relative Pose Prior}. With this binary factor, a known transform $\Delta \ctrans{}{T}{}$
between two pose variables can be specified, with noise $\Sigma$:
\begin{equation}
  p_R(\ctrans{}{T}{A}, \ctrans{}{T}{B}|\Delta\ctrans{}{T}{},\Sigma) = g(\ctransi{}{T}{B}\ctrans{}{T}{A}; \Delta\ctrans{}{T}{}, \Sigma)\ .
\end{equation}
If odometry body pose differences $\Delta \ctrans{}{T}{\mathrm{odom}}(t)$ with noise $\sigma$
are available from e.g. a VIO algorithm running
alongside TagSLAM, a relative pose prior of
$p_R(\ctrans{\mathrm{w}}{T}{\mathrm{body}}(t),
\ctrans{\mathrm{w}}{T}{\mathrm{body}}(t-1)|
\Delta \ctrans{}{T}{\mathrm{odom}}(t),\sigma)$ can be used to insert
the odometry updates into the pose graph.

{\em Tag Projection Factor}. The output of the tag detection
library is a list of tag IDs and the corresponding image
coordinates $\mv{u}_c$ (in units of pixels) of the corners
$c=1\dots 4$ of every tag. This gives rise to one quaternary
tag projection factor per tag:
\begin{equation}
  \begin{split}
    p_T(\ctrans{\mathrm{w}}{T}{\mathrm{body}},
    \ctrans{\mathrm{rig}}{T}{\mathrm{cam}},
    \ctrans{\mathrm{body}}{T}{\mathrm{tag}},
    \ctrans{\mathrm{w}}{T}{\mathrm{rig}}|\{\mv{u}_c\},\sigma_p) =\\
    \prod_{c=1\dots 4}g(\Pi(\ctrans{\mathrm{cam}}{T}{\mathrm{rig}}
    \ctrans{\mathrm{rig}}{T}{\mathrm{w}}
    \ctrans{\mathrm{w}}{T}{\mathrm{body}}
    \ctrans{\mathrm{body}}{T}{\mathrm{tag}}\mv{s}_c);
    \mv{u}_c,\sigma_p)\ .
  \end{split}
\end{equation}
Here, $\mv{s}$ refers to the corner coordinates in the tag reference frame,
i.e. $\mv{s}_1 = [-l/2, -l/2, 0]^\top$,
$\mv{s}_2 = [l/2, -l/2, 0]^\top$,
$\mv{s}_3 = [l/2,  l/2, 0]^\top$,
$\mv{s}_4 = [-l/2, l/2, 0]^\top$
for a tag of side length $l$. A sequence of transforms expresses $\mv{s}$ in
camera coordinates, after which the function $\Pi$ projects \cite{ma2003} the point onto the
sensor plane and converts it to pixel coordinates. The noise parameter $\sigma_p$
is a diagonal matrix that reflects the accuracy of the tag library's corner detector,
which is usually assumed to be about one pixel.

We can visualize the factor structure of Eq.\ (\ref{eq:totalprob}) by means of
a graph as shown in Fig.\ \ref{fig:sample_graph} for the scene from
Fig.\ \ref{fig:scene_with_block}. In Fig.\ \ref{fig:sample_graph},
black squares represent factors, whereas circles denote pose variables
to be optimized.
The prior factors $p_A$ constrain
the static poses  $\ctrans{l}{T}{2}$, $\ctrans{\mathrm{w}}{T}{l}$,
$\ctrans{\mathrm{r}}{T}{\mathrm{c}}$, and
$\ctrans{\mathrm{b}}{T}{105}$. A tag projection factors $p_T$ arising
from an observation of Tag 2 determines the dynamic rig pose
$\ctrans{\mathrm{w}}{T}{\mathrm{r}}(t)$, whereas an observation of Tag
105 likewise yields the block pose
$\ctrans{\mathrm{w}}{T}{\mathrm{b}}(t)$. Assuming that odometry for
the rig is provided by some external algorithm, there is a relative
pose factor $p_R$ connecting the rig poses for $t$ and $t+1$. Two more
tag observations at $t+1$ generate additional factors that further constrain rig and block poses at $t+1$.

\begin{figure}[ht]
  \vspace{0.25cm}
  \newcommand{\relfacgraphsize}{0.75}
  \begin{center}
    \begin{tabular}{cc}
      \resizebox{\relfacgraphsize\columnwidth}{!}{
      \begin{tikzpicture}

\newcommand{\ysepstatic}{0.2cm}  % y sep static poses
\newcommand{\xsepprior}{0.2cm}   % x sep of priors
\newcommand{\xsepdyn}{1.6cm}     % x sep dyn pose
\newcommand{\xsepfac}{1.1cm}     % x sep dyn pose
\newcommand{\nsize}{16mm}        % minimum node size
\newcommand{\facoff}{12mm}       % factor offset
\newcommand{\xsepfactpo}{4.0cm}  % x sep dyn pose t+1
\newcommand{\xseprelfac}{0.75cm} % rel pose prior factor
\tikzstyle{node} = [latent,minimum size=\nsize];

%
%
% The static poses
%
\node[node]                               (lab_T_2)     {$\ctrans{l}{T}{2}$};
\node[node, below=\ysepstatic of lab_T_2]   (w_T_lab)     {$\ctrans{\mathrm{w}}{T}{l}$};
\node[node, below=\ysepstatic of w_T_lab]   (rig_T_cam)   {$\ctrans{\mathrm{r}}{T}{\mathrm{c}}$};
\node[node, below=\ysepstatic of rig_T_cam] (block_T_105) {$\ctrans{\mathrm{b}}{T}{105}$};

%
%
% priors on static poses
%
\factor[left=\xsepprior of lab_T_2]     {pAtag2} {above:{$p_A$}} {lab_T_2} {};
\factor[left=\xsepprior of w_T_lab]     {pAlab}  {above:{$p_A$}} {w_T_lab} {};
\factor[left=\xsepprior of rig_T_cam]   {pAcam}  {above:{$p_A$}} {rig_T_cam} {};
\factor[left=\xsepprior of block_T_105] {pA105}  {above:{$p_A$}} {block_T_105} {};

%
% dynamic poses at t
%
\node[node,right=\xsepdyn of lab_T_2]     (w_T_rig_t)   {$\ctrans{\mathrm{w}}{T}{\mathrm{r}}(t)$};
\node[node,right=\xsepdyn of block_T_105] (w_T_block_t) {$\ctrans{\mathrm{w}}{T}{\mathrm{b}}(t)$};

%
% tag projection factors at t
%
\factor[right=\xsepfac of lab_T_2] {pT2t}       {above:{$p_T$}} {lab_T_2,w_T_lab,rig_T_cam,w_T_rig_t}{};
\factor[right=\xsepfac of block_T_105] {pT105t} {below:{$p_T$}} {block_T_105,w_T_rig_t,rig_T_cam,w_T_block_t}{};

%
% dynamic poses at t+1
%
\node[node,right=\xsepdyn of w_T_rig_t]   (w_T_rig_tp1)   {$\ctrans{\mathrm{w}}{T}{\mathrm{r}}(t+1)$};
\node[node,right=\xsepdyn of w_T_block_t] (w_T_block_tp1) {$\ctrans{\mathrm{w}}{T}{\mathrm{b}}(t+1)$};

%
% tag projection factors at t + 1
%
\factor[right=\xsepfactpo of w_T_lab]   {pT2tp1}  {above:{$p_T$}} {lab_T_2,w_T_lab,rig_T_cam,w_T_rig_tp1}{};
\factor[right=\xsepfactpo of rig_T_cam] {pT105tp1}{below:{$p_T$}} {block_T_105,w_T_rig_tp1,rig_T_cam,w_T_block_tp1}{};

%
% relative pose priors
%

\factor[right=\xseprelfac of w_T_rig_t]  {prelrig}{above:{$p_R$}} {w_T_rig_t,w_T_rig_tp1}{};


\end{tikzpicture}

      }
    \end{tabular}
  \end{center}
  \caption{Factor graph of Eq.\ \ref{eq:totalprob} for the scene shown in Fig.\ \ref{fig:scene_with_block}}
  \label{fig:sample_graph}
\end{figure}

